
\section{Was ist \LaTeX ?}
\input{./preambel/tocframe}
\begin{frame}{Was ist \LaTeX ?}
	\begin{itemize}[<+->]
	\item basiert auf dem Textsatzsystem \TeX
	\item \TeX \mbox{ } bereits 1977 von Donald E. Knuth entwickelt
	\item \LaTeX \mbox{ } = \textbf{La}mport \textbf{TeX}
	\item entwickelt von Leslie Lamport anfang der 80er entwicklet
	\item \LaTeX \mbox{ } stellt eine Vereinfachung von \TeX \mbox{ } dar
	\end{itemize}
\end{frame}

\begin{frame}{Wie funktioniert \LaTeX?}
	\begin{enumerate}[<+->]
	\item \TeX -Datei (.tex)
	\item Compiler + Tools (latex, pdflatex, bibtex, latexmk)
	\item Ausgabedatei (.dvi, pdf)
	\end{enumerate}

	\begin{itemize}[<+->]
	\item TeX Live
	\item MiKTeX
	\end{itemize}
\end{frame}

\section{Syntax \& Aufbau eines \LaTeX -Dokuments}
\input{./preambel/tocframe}
\begin{frame}{Generelle Syntax}
	\begin{itemize}[<+->]
		\item alle Kommandos beginnen mit einem $\backslash$
		\item Kommentare mit einem \%
	\end{itemize}
\end{frame}
\begin{frame}{Beispiel}
	\lstsettex
	\begin{Code}
		\lstinputlisting[linerange=1-19,caption={Teil 1}]{./listings/demonstration.tex}
	\end{Code}

\end{frame}