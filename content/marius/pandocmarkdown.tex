\section{Markdown}

\begin{frame}\frametitle{Ziel}

\begin{itemize}
\item
  \emph{einfache Auszeichnungssprache} zum
\item
  erzeugen von \emph{(X)HTML}
\end{itemize}

\end{frame}

\begin{frame}\frametitle{Ursprung}

\begin{itemize}
\item
  angelehnt an Plaintext-Formatierungen in frühen E-Mails
\item
  basiert auf reStructuredText und Textile.
\end{itemize}

\end{frame}

\begin{frame}[fragile]\frametitle{Beispiel - Code}

\begin{verbatim}
### Beispiel
> A **Markdown**-formatted document should be
publishable as-is, as plain text, 
without looking like it’s been marked up with 
tags or formatting instructions.
 - [John Gruber](http://daringfireball.net/)
\end{verbatim}

\end{frame}

\begin{frame}[fragile]\frametitle{Beispiel - Ergebnis}

\begin{verbatim}
Konsole: markdown beispiel.md > beispiel.html
\end{verbatim}

\begin{block}{Beispiel}

\begin{quote}
A \textbf{Markdown}-formatted document should be publishable as-is, as
plain text, without looking like it's been marked up with tags or
formatting instructions. - \href{http://daringfireball.net/}{John
Gruber}
\end{quote}

\end{block}

\end{frame}

\begin{frame}\frametitle{Technik}

\begin{itemize}
\item
  Textformatierung (fett, kursiv oder als Code)
\item
  Überschriften
\item
  geordnete und ungeordnete Listen
\item
  Links
\item
  Bilder
\item
  direkt HTML-Code
\end{itemize}

\end{frame}

\begin{frame}[fragile]\frametitle{Code}

\begin{verbatim}
`Code-Blöcke`
\end{verbatim}

\texttt{Code-Blöcke}

\end{frame}

\begin{frame}[fragile]\frametitle{Kursiv, Fett und beides}

\begin{verbatim}
*Kursiv* und **Fett**
\end{verbatim}

\emph{Kursiv} und \textbf{Fett}

\end{frame}

\begin{frame}[fragile]\frametitle{Überschriften}

\begin{verbatim}
1. Überschrift
==============

2. Überschrift
--------------

# 1. Überschrift
\end{verbatim}

oder

\begin{verbatim}
#### 4. Überschrift
\end{verbatim}

\begin{block}{4. Überschrift}

\end{block}

\end{frame}

\begin{frame}[fragile]\frametitle{ungeordnete Listen}

\begin{verbatim}
* ungeordnete Liste
* zweiter Eintrag
+ ungeordnete Liste
- ungeordnete Liste
\end{verbatim}

\begin{itemize}
\item
  ungeordnete Liste
\item
  zweiter Eintrag
\item
  ungeordnete Liste
\item
  ungeordnete Liste
\end{itemize}

\end{frame}

\begin{frame}[fragile]\frametitle{geordnete Listen}

\begin{verbatim}
1) geordnete Listen
42) zweiter Eintrag
23) dritter Eintrag
b) geordnete Listen
c) zweiter Eintrag
\end{verbatim}

\begin{enumerate}[1)]
\item
  geordnete Listen
\item
  zweiter Eintrag
\item
  dritter Eintrag
\end{enumerate}

\begin{enumerate}[a)]
\setcounter{enumi}{1}
\item
  geordnete Listen
\item
  zweiter Eintrag
\end{enumerate}

\end{frame}

\begin{frame}[fragile]\frametitle{Links}

\begin{verbatim}
[Websitelink](http://example.com)
\end{verbatim}

\href{http://example.com}{Websitelink}

\end{frame}

\begin{frame}[fragile]\frametitle{Bilder}

\begin{verbatim}
![Beispielbild](pictures/example.png)
\end{verbatim}

\begin{figure}[htbp]
\centering
\includegraphics{pictures/example.png}
\caption{Beispielbild}
\end{figure}

\end{frame}

\begin{frame}[fragile]\frametitle{inline HTML}

\begin{verbatim}
<p>Hello World</p>
\end{verbatim}

Hello World

\end{frame}

\begin{frame}\frametitle{Variation}

\begin{itemize}
\item
  \emph{Google+ Markdown}: reduziertes Markdown nur mit \_kursiv\_,
  *fett* und -durchgestrichen-
\item
  \emph{GitHub Flavored Markdown}: erweitertes Markdown um Features am
  Codeblock, Referenzen und was man sonst so auf Github braucht.
\item
  Multimarkdown und noch einige andere.
\end{itemize}

\end{frame}

\section{Pandoc Markdown}

\begin{frame}\frametitle{Technik}

\begin{itemize}
\item
  Markdown
\item
  Unicode
\item
  Fußnoten
\item
  Tabellen
\item
  Metadaten
\item
  TexMath
\item
  inline Latex
\item
  Haufenweise Optionen (z.B. Inhaltsverzeichnis, Templates,
  Syntax-Highlighting, Style-Referenzen)
\end{itemize}

\end{frame}

\begin{frame}[fragile]\frametitle{Tabellen}

\begin{verbatim}
  Right     Left     Center     Default
-------     ------ ----------   -------
     12     12        12            12
    123     123       123          123
      1     1          1             1
\end{verbatim}

\ctable[pos = H, center, botcap]{rlcl}
{% notes
}
{% rows
\FL
Right & Left & Center & Default
\ML
12 & 12 & 12 & 12
\\\noalign{\medskip}
123 & 123 & 123 & 123
\\\noalign{\medskip}
1 & 1 & 1 & 1
\LL
}

\end{frame}

\begin{frame}[fragile]\frametitle{Metadaten}

z.B.

\begin{verbatim}
% Pandoc Markdown
% Marius Wegner
% 9.12.2012
\end{verbatim}

oder als
\textbf{\href{http://dublincore.org/documents/dces/}{Dublincore}}.

\end{frame}

\begin{frame}[fragile]\frametitle{TexMath}

\begin{verbatim}
$\sqrt[n]{1+x+x^2+x^3+\ldots}$
\end{verbatim}

$\sqrt[n]{1+x+x^2+x^3+\ldots}$

\end{frame}

\begin{frame}[fragile]\frametitle{inline Latex}

\begin{verbatim}
\textbf{hello world}
\end{verbatim}

\textbf{hello world}

\end{frame}

\begin{frame}\frametitle{Nachlesen, Ende}

Zum nachlesen:
\href{http://johnmacfarlane.net/pandoc/README.html}{http://johnmacfarlane.net/pandoc/README.html}

\end{frame}
