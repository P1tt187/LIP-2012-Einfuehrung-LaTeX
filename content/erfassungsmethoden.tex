\section{Erfassungsmethoden}
 \begin{frame}
 	\frametitle{Inhalt}
 	\tableofcontents[%
 		currentsection, % causes all sections but the current to be shown in a semi-transparent way.
% % 		currentsubsection, % causes all subsections but the current subsection in the current section to ...
% % 		hideallsubsections, % causes all subsections to be hidden.
% 		hideothersubsections, % causes the subsections of sections other than the current one to be hidden.
% % 		part=, % part number causes the table of contents of part part number to be shown
% 		pausesections, % causes a \pause command to be issued before each section. This is useful if you
% 		pausesubsections, %  causes a \pause command to be issued before each subsection.
% % 		sections={ overlay specification },
 	]
 \end{frame}
\begin{frame}{Auswahl von Erfassungsmethoden}
	\begin{itemize}[<+->]
	\item Psychomotor Vigilance Task
	\item Number Pairs
	\item n-Back
	\item Compensatory Tracking Task
	\item Visual Analogue Scales
	\end{itemize}
\end{frame}
\begin{frame}{Psychomotor Vigilance Task}
	\begin{itemize}[<+->]
	\item 1982 von Wilkinson \& Houghton vorgestellt
	\item 1985 von Dinges \& Powell bekannt gemacht
	\item Bildschirm zunächst dunkel
	\item nach zufälliger Verzögerungszeit wird dem Probanden ein Reiz Präsentiert
	\item Proband muss innerhalb von fünf Sekunden Reagieren
	\item Reaktionszeit wird gemessen
	\item bei abnehmender Konzentration steigt Reaktionszeit
	\item in Labortests 10 Minuten üblich
	\item bei Handhelds 5 Minuten üblich
	\end{itemize}
\end{frame}
\begin{frame}{Number Pairs}
	\begin{itemize}[<+->]
	\item basiert auf Paradigma von 1985
	\item Proband bekommt Zahlenreihe bestehend aus fünf Ziffern
	\item Proband muss entscheiden ob das zweite und das vierte Element identisch sind
	\item Zeitlimit fünf Sekunden
	\item Reaktionszeit und korrekte Antwort gemessen
	\item 28 Durchläufe üblich
	\end{itemize}
\end{frame}
\begin{frame}{n-Back}
	\begin{itemize}[<+->]
	\item im Jahr 1958 von W. K. Kirchner vorgestellt
	\item Probanden werden aufeinanderfolgende Reize Präsentiert
	\item Proband muss entscheiden ob Reiz mit dem \textit{n}-Elemente vorher übereinstimmt
	\end{itemize}
	\pause
	\begin{block}{Beispiel für n=3}
		T L H C H S \textbf{C} C Q L \textbf{C} K \textbf{L} H C Q T R R K C H R
	\end{block}
	\pause
	\begin{itemize}[<+->]
	\item erfasst wird Reaktionszeit und Antwort des Probanden
	\item bei schlechter Konzentration weniger korrekte Antworten 
	\item Reaktionszeit nimmt zu
	\item üblich 2-Back mit 20 versuchen
	\end{itemize}
\end{frame}
\begin{frame}{Compensatory Tracking Task}
	\begin{itemize}[<+->]
	\item 1996 von Scott Makeig und Keith Jolle vorgestellt
	\item Proband steuert Cursor über Trackball
	\item muss Cursor in Zielmarke halten
	\item Cursor wird in zufällige Richtung abgestoßen
	\item Proband muss Cursor so nahe wie möglich bei Zielmarke halten
	\item mit sinkender Konzentration steigt mittlerer Abstand zum Ziel
	\end{itemize}
\end{frame}