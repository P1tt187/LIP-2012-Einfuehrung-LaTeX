% ich bin ein kommentar 
\documentclass{scrartcl} %brief, artikel, report, buch
%hier ist die preambel, hier wird die allgemeine Formatierung festgelegt
\usepackage[utf8]{inputenc} % zeichenkodierung (wichtig für umlaute)
\usepackage[T1]{fontenc}
\usepackage[ngerman]{babel} %deutsche bezeichner für verzeichnisse
\usepackage{amsmath} %formeln 
\title{Ein Testdokument}
\author{Max Mustermann}
\date{\today}
\begin{document} %hier geht das dokument los 
\maketitle %titelseite
\tableofcontents %inhaltsverzeichnis
\section{Einleitung}
 
Hier kann ein beliebig langer text stehen.       Überflüssige Leerzeichen werden ignoriert
,toll , oder?
 
Ich bin ein neuer absatz.

\subsection{Formeln}
 
\LaTeX{} ist auch ohne Formeln sehr nützlich und
einfach zu verwenden. Grafiken, Tabellen,
Querverweise aller Art, Literatur- und
Stichwortverzeichnis sind kein Problem.
 
Formeln sind etwas schwieriger, dennoch hier ein
einfaches Beispiel.  Zwei von Einsteins
berühmtesten Formeln lauten:
\begin{align}
E &= mc^2                                  \\
m &= \frac{m_0}{\sqrt{1-\frac{v^2}{c^2}}}
\end{align}
Aber wer keine Formeln schreibt, braucht sich
damit auch nicht zu beschäftigen.
\end{document}
